\begin{CJK}{UTF8}{gkai}
\chapter{图像处理}
\end{CJK}


\begin{CJK}{UTF8}{gbsn}


本章中的所有算法都涉及对图像或像素的二维数组执行的操作。通常,图形程序员可能想要更改图像的大小、颜色或其他功能。前三个算法描述了在各种环境下拉伸或缩放图像的技术。第一条强调速度,而第二条强调质量。第四个算法描述了一种使用简化的颜色集显示全彩色图像的方法。


在某些情况下,将多个图像的特征组合起来是很有用的。第七个算法将我们现在熟悉的图像组合代数应用于黑白位图或1位图像。第八个算法讨论了如何有选择地模糊两幅图像,同时结合它们,以模拟相机的光圈景深效果。


有时,期望的结果不是另一个图像,而是图像中某些特征的另一种表示。第五、第六和第九算法描述了从图像中提取区域边界信息的技术。


\newpage
\section{位图快速拉伸}
\begin{center}
\small{
Tomas M\"oller\\
Lund Institute of Technology 
Hoganas, Sweden}
\end{center}

\subsection*{介绍}

这里提出了一个整数算法,用于将位图的任意水平线或垂直线拉伸到任何其他任意直线上。该算法可用于要求接近实时性或实时性的绘图和绘图程序。应用程序区域的例子包括扩大和缩小位图的矩形区域,以及将矩形区域包装到圆形区域上。
\subsection*{算法}
程序本身非常简单,大多数计算机图形程序员可能都熟悉它所基于的Bresenham划线算法(1965)。事实上,它可以基于任何画线算法;然而,Bresenham被选中,因为它是基于整数的,并且在计算机图形社区中非常普遍。对于那些不熟悉Bresenham算法的人来说,伪代码用于在第一个八边形中绘制线段。

\IncMargin{1em}
\begin{algorithm} 
\SetKwData{Left}{left}
\SetKwData{This}{this}
\SetKwData{Up}{up} 
\SetKwFunction{Union}{Union}
\SetKwFunction{WritePixel}{WritePixel}
\SetKwFunction{FindCompress}{FindCompress} 
\SetKwInOut{Input}{input}
\SetKwInOut{Output}{output}

\SetKwData{EdgeSet}{EdgeSet}
\SetKwData{Point}{point}
	
	\Input{由点(x1,y1)和点(x2,y2)构成的线段} 
	\Output{将线段绘制到位图上}
	 \BlankLine 
	 \BlankLine
	 
	 
	 
	 \tcp{从(x1, y1)到(x2, y2)在第一个八边形内画一条线;
所有变量都是整数}
	dx $\leftarrow$ x2 - x1\\
	dy $\leftarrow$ y2 - y1\\
	e $\leftarrow$ 2 * dy - dx\\
	\For{i$\leftarrow$ 1,i$\leqslant$ dx, i$\leftarrow$ i+1}{
		\WritePixel(x1,y1)	\tcp{在图上显示(x1,y1)}
		\While{e$\geqslant$ 0}{
			y1 $\leftarrow$ y1 + 1\\
			e $\leftarrow$ e - 2 * dx\\
		}
		x1 $\leftarrow$ x1 + 1\\
		e $\leftarrow$ e + 2 * dy
	}
	 
	 
	 
 	 	  \caption{Line(x1,y1,x2,y2)}
 	 	  \label{algo:Line} 
 	 \end{algorithm}
\DecMargin{1em} 

上面的伪代码也适用于第二个八分位数,但是在这种情况下,行不会是连续的,因为x1总是递增1。这非常适合算法。


让我们回到拉伸算法的解释上。x1和y1不能被解释为二维直线上的一对坐标,它们必须被解释为一维坐标。dx必须解释为目标线的长度,dy是源线的长度。使用这些解释,x1是目标线上的坐标,y1是源线上的坐标。对于目标线上的每个像素,从源线上选择一个像素。这些像素以统一的方式选择。参见图\ref{fig:drawline}。
\begin{figure}[htbp]%[htbp]
  \centering
  \includegraphics[totalheight=2in]{./fig/1.png}
  \caption{源线与目标线} 
  \label{fig:drawline}
\end{figure}

如果$dx$大于$dy$,那么目标线比源线长。因此,在绘制到目的线上时,源线将被放大。另一方面,如果$dy$大于$dx$,源线就会减小。如果$dx = dy$,算法会得到与源相同的直线。下面是伪代码中完整的stretcher算法,重写后能够处理$x2 < x1$和$y2 < y1$的行。

\IncMargin{1em}
\begin{algorithm} 
\SetKwData{Left}{left}
\SetKwData{This}{this}
\SetKwData{Up}{up} 
\SetKwFunction{Union}{Union}
\SetKwFunction{WritePixel}{WritePixel}
\SetKwFunction{Abs}{abs}
\SetKwFunction{Sign}{sign}
\SetKwFunction{ReadPixel}{ReadPixel}
\SetKwFunction{SetColor}{SetColor}
\SetKwFunction{FindCompress}{FindCompress} 
\SetKwInOut{Input}{input}
\SetKwInOut{Output}{output}

\SetKwData{EdgeSet}{EdgeSet}
\SetKwData{Point}{point}
	
	\Input{} 
	\Output{}
	 \BlankLine 
	 \BlankLine
	 
	 
	 
	 \tcp{从(x1, y1)到(x2, y2)在第一个八边形内画一条线;
所有变量都是整数}
	\tcp{将源线(y1 ~ y2)延伸到目标线(x1 ~ x2)。}
	\tcp{源线和目标线都是水平的}
	\tcp{yr =从其中读取像素的水平线}
	\tcp{yw =要写入像素的水平线}
	\tcp{ReadPixel(x, y)返回像素(x, y)处的颜色}
	\tcp{WritePixel(x, y) 用当前颜色写入(x, y)处的像素}
	\tcp{SetColor(Color) 设置当前写入的颜色为Color}
	dx $\leftarrow$ \Abs(x2 - x1)\\
	dy $\leftarrow$ \Abs(y2 - y1)\\
	sx $\leftarrow$ \Sign(x2 - x1)\\
	sy $\leftarrow$ \Sign(y2 - y1)\\
	e $\leftarrow$ 2 * dy - dx\\
	dx2 $\leftarrow$ 2 * dx\\
	dy $\leftarrow$ 2 * dy\\
	\For{i$\leftarrow$ 1,i$\leqslant$ dx, i$\leftarrow$ i+1}{
		color $\leftarrow$ \ReadPixel(x, y)\\
		\SetColor(Color)\\
		WritePixel(x1, yw)
		
		\While{e$\geqslant$ 0}{
			y1 $\leftarrow$ y1 + sy\\
			e $\leftarrow$ e - dx2\\
		}
		x1 $\leftarrow$ x1 + 1\\
		e $\leftarrow$ e + 2 * dy
	}
	 
	 
	 
 	 	  \caption{Stretch(x1, y1, x2, y2, yr, yw)}
 	 	  \label{algo:Stretch} 
 	 \end{algorithm}
\DecMargin{1em} 

如果$x$等于0,那么符号函数不需要返回0,因为$dx$或$dy$都等于0,这意味着一条长度为1的直线。由于该算法只使用整数运算,而不使用乘法或除法,因此非常高效和快速。


这个小程序的另一个有趣之处是,它可以用来生成几种不同形状的位图。下面列出了一些可以用来渲染的东西。
\subsection*{一些项目使用位图扩展器}
\begin{itemize}
	\item 包裹在圆形或椭圆形区域上的矩形图片。关于绕圈,请参阅附录中的源代码。
	\item 放大和缩小位图的矩形部分。参见附录中的源代码。
	\item 将位图的矩形部分绕平行梯形包绕。例如,一个绕x或y轴旋转,然后进行透视转换的矩形可以用作目标形状。
\end{itemize}


\subsection*{进一步的工作}
为了改进算法,也许可以添加一个抗锯齿例程.\\
See also G1, 147; G1, 166; G3, A.2.




\newpage
\section{一般的滤波图像缩放}
\begin{center}
\small{
Dale Schumacher\\
St. Paul, Minnesota}
\end{center}


栅格图像可以看作是连续二维函数$f(x, y)$的样本的矩形网格。这些样本被假定为连续函数在给定样本点的精确值。理想的光栅图像缩放程序包括重建原始的连续函数,然后以不同的速率重新采样该函数(Pratt, 1991;Foley $ et al$., 1990 )。采样率越高(采样越靠近),采样就越多,图像也就越大。采样率越低(采样间隔越远)产生的样本越少,因此图像越小。幸运的是,我们不需要真正地重建整个连续函数,而只是确定重建函数在与新样本对应的点上的值,这是一个更容易的任务(Smith, 1981)。仔细选择过滤器,这个重采样过程可以分两步进行,首先水平地拉伸或缩小图像,然后垂直地拉伸或缩小(反之亦然),可能有不同的比例因子。双通道方法的运行时成本$O$(image\_width*image\_height* (filter\_width + filter\_height)比简单的二维过滤$O$(image\_width*image\_height*filter\_width*filter\_height)要低得多。

\begin{figure}[htbp]%[htbp]
  \centering
  \includegraphics[totalheight=1.7in]{./fig/1.2.1.png}
  \caption{$sinc$示例} 
  \label{fig:sinc}
\end{figure}

放大图像的过程有很多名字,包括放大、拉伸、缩放、插值和上采样。我将把这个过程称为放大。缩小图像的过程也有很多名字,包括缩小、缩小、按比例缩小、抽取和下采样。我将把这个过程称为简化。这些过程将在一维而不是二维中进行解释,因为缩放是在每个轴上独立进行的。

在放大过程中,我们通过应用滤波函数来确定每个源像素对每个目标像素的贡献。采样理论表明,$sinc$函数$f(x) = \sin(πx)/πx$是理想的重构函数;然而,我们有一个有限的样本集,并且需要一个具有有限支持的过滤器(即过滤器非零的区域)。我在这个例子中使用的过滤器是一个三次函数,$f(x) = 2|x| 3 - 3|x| 2 +1$,从- 1到+1,当单独应用时,它覆盖了每个样本的单位体积。图1比较了这些过滤函数。重采样滤波器的设计是一个没完没了的争论的来源,超出了这个宝石的范围,但在许多其他作品中讨论(Pratt, 1991; Turkowski, 1990; Mitchell, 1988; Smith, 1982; Oppenheim and Schafer, 1975; Rabiner and Gold, 1975).。为了应用滤镜,我们将滤镜函数的副本放在每个源像素的中心,并缩放到该像素的高度。对于每个目标像素,我们计算源图像中相应的位置。我们将这一点上加权滤波函数的值相加,以确定目标像素的值。图\ref{fig:sinc2}说明了这个过程。

\begin{figure}[htbp]%[htbp]
  \centering
  \includegraphics[totalheight=4in]{./fig/1.2.2.png}
  \caption{滤波示例} 
  \label{fig:sinc2}
\end{figure}

在缩小过程中,过程是相似的,但不完全相同,因为我们必须考虑频率混叠。采样理论将奈奎斯特频率定义为能够正确捕获连续源信号中所有频率成分的采样率。奈奎斯特频率是源信号中最高频率分量频率的两倍。任何频率成分高于采样率的一半将被不正确地采样,并将被混叠到一个更低的频率。

因此,重构信号将只包含采样率的一半或更少的频率成分。在放大过程中,我们拉伸重构信号,降低其分量频率。然而,在缩小过程中,我们正在缩小重构信号,提高其分量频率,并可能超过我们新的采样率的奈奎斯特频率。为了创建合适的样本,我们必须消除重采样奈奎斯特频率以上的所有频率成分。这可以通过图像缩小因子拉伸滤波函数来实现。此外,由于每个源像素处的滤波器更宽,和将按比例更大,并应除以相同的因子进行补偿。图\ref{fig:sinc3}说明了这个过程。

\begin{figure}[htbp]%[htbp]
  \centering
  \includegraphics[totalheight=4in]{./fig/1.2.3.png}
  \caption{滤波示例} 
  \label{fig:sinc3}
\end{figure}

到目前为止,我们只考虑了一维情况。我们将其扩展到典型的光栅图像的二维情况,首先水平缩放,然后垂直缩放。这里将不再演示缩放最小目标轴的进一步优化。滤波操作会导致大量的计算,所以我们尽可能多地预先计算。每个行(或列)的缩放过程是相同的。过滤器的位置和面积是固定的;这样,我们就可以预先计算出每个目标像素的贡献者和相应的滤波器权值。计算目标像素贡献者的伪代码如\ref{algo:calculate_contributions}。

\IncMargin{1em}
\begin{algorithm} 
\SetKwData{Left}{left}
\SetKwData{This}{this}
\SetKwData{Up}{up} 
\SetKwFunction{Union}{Union}
\SetKwFunction{WritePixel}{WritePixel}
\SetKwFunction{Abs}{abs}
\SetKwFunction{Floor}{floor}
\SetKwFunction{Ceiling}{ceiling}
\SetKwFunction{Sign}{sign}
\SetKwFunction{ReadPixel}{ReadPixel}
\SetKwFunction{SetColor}{SetColor}
\SetKwFunction{FindCompress}{FindCompress} 
\SetKwFunction{Filter}{filter}
\SetKwFunction{AddContributor}{add\_contributor}

\SetKwInOut{Input}{input}
\SetKwInOut{Output}{output}

\SetKwData{EdgeSet}{EdgeSet}
\SetKwData{Point}{point}
	
	\Input{} 
	\Output{}
	 \BlankLine 
	 \BlankLine
	 
	 
	 
	 %\tcp{从(x1, y1)到(x2, y2)在第一个八边形内画一条线;所有变量都是整数}
	
	scale $\leftarrow$ dst\_size/src\_size;\\
	center $\leftarrow$ destination/scale;\\
	\If{scale < 1.0}{
		width $\leftarrow$ filter\_width/scale;\\
		fscale $\leftarrow$ 1.0/scale;\\
	}
	\lElse{
		width $\leftarrow$ filter\_width;\\
		fscale $\leftarrow$ 1.0;\\	
	}
	left $\leftarrow$ \Floor(center - width);\\
	right $\leftarrow$ \Ceiling(center + width);\\
	\For{source $\leftarrow$ left,source $\leftarrow$ source + 1, source $\leqslant$ right}{
		weight ← \Filter((center – source)/fscale)/fscale;
		\AddContributor(destination, source, weight);
	}
	 
	 
	 
 	 	  \caption{calculate\_contributions(destination)}
 	 	  \label{algo:calculate_contributions} 
 	 \end{algorithm}
\DecMargin{1em} 

在计算出贡献之后,目标图像的所有行(或列)都可以使用相同的预计算的过滤器值进行处理。下面的伪代码\ref{algo:scale_row}显示了这些值用于扩展单个目标行的应用程序。

\IncMargin{1em}
\begin{algorithm} 
\SetKwData{Left}{left}
\SetKwData{This}{this}
\SetKwData{Up}{up} 
\SetKwFunction{Union}{Union}
\SetKwFunction{WritePixel}{WritePixel}
\SetKwFunction{Abs}{abs}
\SetKwFunction{Floor}{floor}
\SetKwFunction{Ceiling}{ceiling}
\SetKwFunction{Sign}{sign}
\SetKwFunction{ReadPixel}{ReadPixel}
\SetKwFunction{SetColor}{SetColor}
\SetKwFunction{FindCompress}{FindCompress} 
\SetKwFunction{Filter}{filter}
\SetKwFunction{AddContributor}{add\_contributor}

\SetKwInOut{Input}{input}
\SetKwInOut{Output}{output}

\SetKwData{EdgeSet}{EdgeSet}
\SetKwData{Point}{point}
	
	\Input{} 
	\Output{}
	 \BlankLine 
	 \BlankLine
	 
	 
	 
	 %\tcp{从(x1, y1)到(x2, y2)在第一个八边形内画一条线;所有变量都是整数}
	
	\For{ i ← 0, i ← i + 1, i < dst\_size }{
		v ← 0;\\
		\For{ j ← 0, j ← j + 1, j < contributors[i] }{
			s ← contributor[i][j];\\
			w ← weight\_value[i][j];\\
			v ← v + (source\_row[s]*w);\\
		}
		destination\_row[i] ← v;\\
	}
	
 	 	  \caption{scale\_row(destination\_row, source\_row)}
 	 	  \label{algo:scale_row} 
 	 \end{algorithm}
\DecMargin{1em} 


\begin{figure}[htbp]%[htbp]
  \centering
  \includegraphics[totalheight=2in]{./fig/1.2.4.png}
  \caption{滤波示例} 
  \label{fig:sinc4}
\end{figure}

\begin{figure}[htbp]%[htbp]
  \centering
  \includegraphics[totalheight=2in]{./fig/1.2.5.png}
  \caption{滤波示例} 
  \label{fig:sinc5}
\end{figure}

\begin{figure}[htbp]%[htbp]
  \centering
  \includegraphics[totalheight=2in]{./fig/1.2.6.png}
  \caption{滤波示例} 
  \label{fig:sinc6}
\end{figure}

\begin{figure}[htbp]%[htbp]
  \centering
  \includegraphics[totalheight=2in]{./fig/1.2.7.png}
  \caption{滤波示例} 
  \label{fig:sinc7}
\end{figure}

\begin{figure}[htbp]%[htbp]
  \centering
  \includegraphics[totalheight=2in]{./fig/1.2.8.png}
  \caption{滤波示例} 
  \label{fig:sinc8}
\end{figure}

然后将相同的过程应用于图像的列—首先根据垂直比例因子(可能与水平比例因子不同)预计算过滤器的贡献,然后处理从中间(水平比例)图像到最终目标图像的列。

在附录中提供的源代码中,给出了许多过滤器函数,可以很容易地添加新的函数。zoom()函数以所需过滤器的名称和过滤器支持作为参数。图\ref{fig:sinc4}到图\ref{fig:sinc8}显示了不同滤波器对样本图像的影响,以及每个滤波器函数的脉冲响应图。

样本图像在两个方向上都按比例放大了12倍。图\ref{fig:sinc4}显示了一个框过滤器,它相当于直接复制像素值,因为它显示了相当多的平铺或“锯齿”。图\ref{fig:sinc5}显示了三角形或Bartlett滤波器,相对于方框来说是一个相当大的改进,计算仍然简单,但仍然有明显的过渡线。图\ref{fig:sinc6}显示了一个三次b样条,它没有产生尖锐的过渡,但是它的宽度导致了过度的模糊。三角形和b样条函数分别通过盒形滤波器与自身卷积1次和3次来计算。图\ref{fig:sinc7}显示了Lanczos 3滤波器,一个在$-3$到$+3$范围外衰减为零的$sinc$函数,它显示了在完全sinc函数时出现的过度的“振铃”效应。图\ref{fig:sinc8}显示了Mitchell滤波器($B = \frac{1}{3}, C =\frac{1}{3} $),一个没有急剧过渡的三次函数,以及“振铃(ringing)”和“模糊(blurring)”效果之间的一个很好的折衷。

See also G1, 147; G1, 166; G3, A.1.

\newpage
\section{位图缩放操作的优化}
\begin{center}
\small{
Dale Schumacher\\
St. Paul Minnesota}
\end{center}

这一节描述了一系列位图缩放操作的优化。我们没有给出一般的缩放算法,而是利用了几个特定于应用程序的限制,这些限制允许显著减少执行时间:每像素图像的位、已知的源和目标位图大小、以及位压缩的水平光栅存储和显示格式。示例应用程序是在典型的视频监视器上显示传真位图图像。

我们首先假设在内存中有源FAX,未压缩,存储为8位字节,每个字节的高阶位表示沿水平行的一组8个字节中最左边的像素。此外,在选择示例缩放因子时,我们假设源FAX的分辨率在两个方向上都是200点每英寸。如果数据是常用的200 × 100 dpi格式,我们可以通过复制每个扫描线使其为200 × 200 dpi,这是我们在解压阶段经常可以处理的任务。最初,我们将假设数据存储为与显示所用的位值匹配的白色和黑色位值。下面将讨论一种反演0位和1位含义的好方法。最后,我们假设目标位图的格式与源位图相同。

由于我们的示例图像分辨率高于您的典型视频监视器,我们将只考虑缩小图像的情况,而不是放大它。同样,我们用8来表示比例因子,$ \frac{7}{8}=87.5\%$,$ \frac{6}{8}=75\%$,$\frac{5}{8}=62.5\%$,$ \frac{4}{8}=50\%$,$ \frac{3}{8}=37.5\%$,$\frac{2}{8}=25\% $。一般算法工作如下:从源图像中取一条扫描线。对于每个字节,使用字节值作为查找表的索引,该查找表给出给定输入字节的简化位。将派生的输出位移位到累加器中。每个输入字节加到累加器的位数是基于比例因子的(例如,如果我们减少到$ \frac{5}{8} $比例,我们为每个8位输入生成5位输出)。当累加器中至少有8位时,我们从累加器中删除最左边的8位,并将它们作为输出字节写入目标扫描行。扫描线末端剩余的任何位都会被移到相应的位置并输出。许多源扫描线可以完全跳过,再次基于比例因子(例如,在$ \frac{5}{8} $的比例下,我们每8个扫描线中只处理5个,跳过3个)。

\begin{figure}[htbp]%[htbp]
  \centering
  \includegraphics[totalheight=2.3in]{./fig/1.3.1.png}
  \caption{缩放操作} 
  \label{fig:1.3.1}
\end{figure}

既然已经理解了基本的算法,我们可以讨论一些有用的变化和改进过程。该算法的核心是约简查找表。如果我们需要反转最终图像中的黑白图像,一种方法是反转查找表中存储的位。然后,它将映射到1111b,而不是$00000000b$映射到$00000b$。这本质上给了我们在缩放期间免费的光度反演。类似地,通过仔细创建查找表,我们可以解决另一个问题,同样是免费的。如果我们缩小到$\frac{3}{8}$的比例,我们将在每8位中寻找3个来输出。图\ref{fig:1.3.1}a显示了这样做的最简单的方法。一个更好的方法是模拟一种对源比特进行滤波或加权平均的形式,如图\ref{fig:1.3.1}b所示。由于查找表可以在编译时创建,所以使用更复杂的算法创建表的计算成本与运行时性能无关。为了进行适当的过滤缩放,我们真的应该跨相邻扫描线和跨字节边界应用过滤器。由于这些操作会带来很高的运行时成本,并且以有限的方式应用过滤,我们可以在没有额外成本的情况下展示改进,所以我们做的是便宜的事情。即使在这些限制条件下,使用滤波也比直接对输入进行子采样要好,如图\ref{fig:1.3.1}a所示。

您喜欢的任何类型的传递函数都可以以同样的方式应用,在8位跨度的限制内,并且仅以黑白作为输入值。您甚至可以做一些事情,如反转位的顺序,这可以与不同的存储顺序一起使用,以使图像从左到右翻转,或将其旋转180度(以防有人将图像倒过来输入扫描仪)。将表扩展到16位,这将占用128Kb的内存,而不是8位表使用的256b,从而提供了更大的灵活性。使用16位时,您可以使用更大的跨度,并且可以选择16倍而不是8倍的比例因数,这可能会使您的视频显示尺寸更匹配。这些技术以及附录中给出的示例代码只是简单的构建块。检查您自己的应用程序的约束,以找到更多应用这些原则和提高代码性能的方法。

See also G1, 147; G1, 166; G2, 57; G2, 84.



\newpage
\section{一个简单的减色滤镜}
\begin{center}
\small{
Dennis Bragg\\
Graphics Software Inc.\\
Bullard, Texas}
\end{center}
\subsection*{介绍}
提出了一个简单的滤波器,将一个24位的彩色光栅图像减少到15位有效位,并消除了可见的颜色步进问题。所得到的图像可以直接显示在16位帧缓冲器上,或用作颜色量化方法的输入,以进一步减少图像中的颜色数量。

光栅图像通常存储为一个24位像素数组,8位分配给每个红、绿、蓝(RGB)组件。每个RGB组件包含256种可能的强度级别。图1(见彩色插入图)是一个24位图像,使用了2215种不同的颜色。注意彩色球平滑连续的阴影。

不幸的是,能够显示24位彩色图像的帧缓冲区并不总是可用的。使用8位像素作为256色彩色地图索引的彩色显示器被广泛使用。颜色量化方法(Gervautz和Purgathofer, 1990)常用于减少24位图像中使用的颜色数量,以便在8位设备上准确显示。

每像素可以显示16位颜色的帧缓冲器(每个RGB组件5位,加上一个属性位)也变得越来越便宜。在16位帧缓冲区上显示24位图像的典型解决方案是屏蔽每个RGB组件的三个最不重要的位。这种方法将每种颜色的256个强度级别减少到只有32个级别。

一个发生在减色平滑阴影图像的问题是颜色步进。在原始24位图像中,亮度从暗到亮连续变化的区域,在16位或8位帧缓冲区中显示时,通常会出现明显的亮度级步骤。在第2页(见彩色插入图)中,第1页的图像使用Gervautz和Purgathofer的颜色量化方法减少到256色。由于可供选择的颜色数量有限,请注意踏在球上的颜色。

该宝石通过一个加权随机量来改变每个像素RGB组件的强度级别,从而解决了颜色步进问题。方差量的加权方式这种方法的结果图像中任何像素局部区域的平均值非常接近源图像的实际24位颜色。

结果图像每像素包含15个有效的颜色位,每个RGB组件包含5个有效的颜色位。图像可以直接显示在16位帧缓冲区上,或用作颜色量化方法的输入,以进一步减少颜色的数量。得到的图像有一些“颗粒”的外观,但比可见的颜色梯度要少得多。

\subsection*{滤波器}

过滤器分别考虑每个像素的RGB组件。将一个分量的256个强度等级划分为32个相等的区域。每个区域覆盖8个强度等级。第一个区域的强度等级为0,下一个区域的强度等级为8,以此类推。

RGB组件的强度将被设置为这些区域之一。如果将组件设置为最接近的强度级别,得到的图像仍然会显示颜色步进。相反,强度除以8(或模量)的余数被确定。这给出了一个从0到7的数字。生成一个范围为0到8的随机数,并与余数进行比较。如果余数小于或等于随机数,则分量强度增加8。这具有以一种随机的方式改变组件的效果,但偏重于最接近的强度水平。

接下来,根据用户提供的噪声水平,将一些随机噪声添加到组件强度中。噪音的添加消除了任何残留的色彩踏步,否则可能是显而易见的。最后,组件较低的3位被屏蔽,将每像素的有效位数减少到15。

这个过程产生的RGB组件与原始的24位组件有很大的不同。然而,图像任意局部区域的像素分量的平均强度与原始图像的平均强度非常接近。在第三张图中(见彩色插入图),首先对原始的24位图像进行滤波处理,然后采用与图2相同的方法将图像压缩到每像素8位。
\subsection*{实现}
这个过滤器是用函数rgbvary()实现的。该函数需要四个参数:一个由待处理像素的RGB组件组成的三个字符数组(RGB),一个指定所需噪声级别的整数(noise\_level),以及像素的$x$和$y$位置($x$和$y$)。

该函数返回源RGB数组中修改后的RGB组件。噪音等级可以从0(无噪音)到8(吵闹!)2级的噪音在实践中效果很好。

像素的x和y位置由两个宏(jitterx和jittery)使用,它们生成随机数。抖动宏基于GRAPHICS GEMS中的抖动函数(Cychosz, 1990)。使用抖动的优点是它总是在特定的$x$, $y$位置上以相同的幅度变化一个像素。当你在动画中减少几帧的色彩时,这是很重要的。使用标准的随机数生成器将在动画播放时产生“雪花”效果。jitter函数消除了这个问题。

在调用rgbvary()以初始化jitter宏使用的查找表之前,必须调用函数jitter\_init()。这个过程使用标准的C函数rand()来填写表格。

\subsection*{总结}

一种滤波器被提出,以减少24位图像为15有效位每像素。该程序消除了颜色步进的问题,但代价是外观略有颗粒。生成的图像可以直接显示在16位帧缓冲上,或用作颜色量化方法的输入,以进一步减少颜色。

\begin{figure}[htbp]%[htbp]
  \centering
  \includegraphics[totalheight=5in]{./fig/1.4.png}
  \caption{颜色滤波} 
  \label{fig:1.4}
\end{figure}

\newpage
\section{从抽样数据得到的紧凑等高线}
\begin{center}
\small{
Dennis Bragg\\
Graphics Software Inc.\\
Bullard, Texas}
\end{center}


\subsection*{问题}
包括医学成像、地震学和气象学在内的许多领域的数据,都是在一个大立方体网格的顶点上采集的一组测量数据。在这些领域中,从数据立方体生成可视化表示的技术非常重要。许多常见的可视化技术将数据值视为连续函数$F$的样本函数值,并对某些$c$生成$F(x, y, z) = c$的分段平面近似,即函数的等值线。最初的《图形瑰宝》之一,“从采样数据定义表面”,调查了几种最著名的从数据立方体生成等高线的技术(Hall, 1990)。

在本文中,我们将对该类型的所有技术进行增强。这种增强减少了任何等高线近似的元素数量,并改善了元素的形状。第一个改进通常将表示的大小减少约$50\%$,允许更快的重新显示并减少内存需求。第二种方法通过避免在许多照明模型中造成不希望看到的阴影影响的狭窄元素,从而得到质量更好的图片。

\subsection*{基于立方体的轮廓}
一些作者提出了大致类似的方法,可以从立方体数据网格中创建可视化的等高线。这些方法分别处理每个立方体上的数据,并利用沿立方体边缘的线性插值来计算位于等高线上的点集合。在Lorenson和Cline的Marching Cubes算法(Lorenson和Cline, 1987)中,这些交点通过表查找来连接成边和三角形,表查找的基础是定义立方体顶点的值$F(x, y, z) - c$的符号。

不幸的是,这种方法并不能保证轮廓的连续性,因为共享一张带有混合符号的脸的相邻立方体可以以不同的方式划分(Durst, 1988)。其他人提出了另一种方法,通过对模糊人脸中心的函数进行采样来消除这种情况的歧义(Wyvil等人,1986)。我们称之为这样的方法,通过沿着三次网格边缘的线性插值计算出轮廓的顶点,基于边缘的插值方法。

基于边缘的插值方法的另一个问题是,它们产生的表面网格可能非常不规则,即使是简单的三元数据。这些不规则性包括微小的三角形(当轮廓通过立方体网格的顶点时产生)和狭窄的三角形(当轮廓通过网格的边缘时产生)。根据我们的经验,在某些曲面网格中,这种三角形可以占到三角形的$50\%$。这些形状糟糕的元素通常会降低渲染算法和有限元分析应用于网格的性能,而对近似的总体精度贡献很小。

\subsection*{紧凑立方体}
这个章节的贡献是一种通用的技术,用于从基于边的插值中消除近简并三角形的问题。该技术背后的想法很简单:当网格的一个顶点靠近表面时,将网格“弯曲”一点,这样顶点就位于表面上。小三角形坍缩成点,小三角形坍缩成边,只剩下形状良好的大三角形。本文的其余部分概述了这一理念的实现;有更详细的解释(Moore和Warren, 1991)。

对数据立方体应用任何基于边的插值算法,并在此过程中,记录沿着立方体边缘生成的每个顶点,该顶点附近的立方体网格的点。我们称这个顶点为它最近的网格点的卫星。如果顶点位于一条边的中点,则可以使用这条边的任意端点,只要共享这条边的所有其他立方体使用同一个端点。当算法的这一阶段完成后,就得到了等高线的三角剖分S和一个离三角剖分的每个顶点最近的网格点。
\begin{figure}[htbp]%[htbp]
  \centering
  \includegraphics[totalheight=1.8in]{./fig/1.5.1.png}
  \caption{二维案例表紧凑型立方体} 
  \label{fig:1.5.1}
\end{figure}

\begin{figure}[htbp]%[htbp]
  \centering
  \includegraphics[totalheight=3in]{./fig/1.5.2.png}
  \caption{一个二维立方体的例子} 
  \label{fig:1.5.2}
\end{figure}



为了得到一个新的、更小的等值线近似值,请采用以下步骤(算法\ref{algo:approximation} ):
\IncMargin{1em}
\begin{algorithm} 
\SetKwData{Left}{left}
\SetKwData{This}{this}
\SetKwData{Up}{up} 
\SetKwFunction{Union}{Union}
\SetKwFunction{WritePixel}{WritePixel}
\SetKwFunction{Abs}{abs}
\SetKwFunction{Floor}{floor}
\SetKwFunction{Ceiling}{ceiling}
\SetKwFunction{Sign}{sign}
\SetKwFunction{ReadPixel}{ReadPixel}
\SetKwFunction{SetColor}{SetColor}
\SetKwFunction{FindCompress}{FindCompress} 
\SetKwFunction{Filter}{filter}
\SetKwFunction{AddContributor}{add\_contributor}

\SetKwInOut{Input}{input}
\SetKwInOut{Output}{output}

\SetKwData{EdgeSet}{EdgeSet}
\SetKwData{Point}{point}
	
	%\Input{} 
	%\Output{}
	 \BlankLine 
	 \BlankLine
	 
	 
	 
	 %\tcp{从(x1, y1)到(x2, y2)在第一个八边形内画一条线;所有变量都是整数}
	
	\For{ each triangle $T$ in $S$ }{
		\If{the vertices of $T$ are satellites of distinct gridpoints}{
			produce a triangle connecting the gridpoints\;
		}
		\Else{
			$T$ collapses to a vertex or edge so ignore it\;
		}
		\For{ each gridpoint $g$ of the new triangulation }{
			displace $g$ to the average position of its satellites;
		}
	}
	
 	 	  \caption{等值线近似值}
 	 	  \label{algo:approximation} 
 	 \end{algorithm}
\DecMargin{1em} 

\begin{figure}[htbp]%[htbp]
  \centering
  \includegraphics[totalheight=1.8in]{./fig/1.5.3.png}
  \caption{球面的两个近似值} 
  \label{fig:1.5.3}
\end{figure}

\begin{figure}[htbp]%[htbp]
  \centering
  \includegraphics[totalheight=1.8in]{./fig/1.5.4.png}
  \caption{两个股骨头的近似图} 
  \label{fig:1.5.4}
\end{figure}

该方法的第一步是定义新网格连接点的拓扑结构。在S中,一个特定网格点的所有卫星被合并成结果网格中的单个顶点。因此,当一个网格点被“切掉”时产生的小三角形被折叠到网格点上。当两个顶点非常接近同一个网格点时产生的窄三角形被折叠成一条边。图1以两个维度说明了这一点。从这个角度来看,如果原曲面网格是连续的,那么算法第一步生成的网格也必须是连续的。

在第二步中,网格的顶点被平移到原始等高线上或附近。由于每个新的顶点位置被选择为位于原始轮廓上一小簇点的平均位置,新的近似通常只与原始轮廓稍微偏离。

图\ref{fig:1.5.2} 说明了这种方法应用于二维网格。上面的部分说明了第一步的结果。下面的部分说明了第二步的输出。图形上部的短边已折叠成下部的顶点。

实际上,这种方法效果很好,将三角形的数量减少了$40\%$到$60\%$。图\ref{fig:1.5.3} 显示了一个球体由游行立方体(a)和相同的球体后紧凑的立方体(B)的应用。图\ref{fig:1.5.4}显示了一个人类的股骨,最初作为CT数据,为波状外形的游行立方体(a)和紧凑的立方体(B)。在每个例子中,三角形的数量减少了使用紧凑的数据集,其余三角形的形状也得到了明显的改善。

正如这里所描述的,紧凑立方体产生的轮廓可能有几个不受欢迎的特征。首先,最终轮廓的边界可能不在定义的立体网格的边界上。第二,通过公共网格点附近的轮廓的两个不相交的薄片可以在该网格点融合。Moore和Warren(1991)描述了对Compact Cubes的简单修改,解决了这些问题。


See also G1, 552; G1, 558; G2, 202.

\newpage
\section{从位图生成等值线}
\begin{center}
\small{
Tim Feldman\\
Island Graphics Corporation\\
San Rafael, California}
\end{center}


本节提出了一种算法,该算法遵循采样数据数组中的轮廓边缘。它使用Freeman链编码生成一个向量列表,描述轮廓的轮廓。

假设您有一个已采样或“数字化”成灰度像素矩形阵列的地形图。不同的像素值对应不同的地形高程。该算法可用于生成以等高线表示地形高程的“地形图”。附录中给出了一个遵循一条轮廓线的示例程序(\textit{contour.c})。

该算法能够处理包含单个样本点的等高线、围绕不同海拔区域的等高线、不形成闭合曲线的等高线、以及形成交叉曲线、形成环路的等高线。在所有情况下,它遵循轮廓的最外层边缘。给定阵列中高程轮廓的初始点,该算法找到轮廓的边缘。然后顺时针方向沿着这条边走,直到它回到起点。沿着每个方向向量描述了从路径上的一个像素到路径上的下一个像素的方向。路径上的所有像素都是近邻。因此,向量可以被认为是传统二维向量的方向部分,其长度部分总是等于一个像素。这样的向量列表被称为“弗里曼链”,以其创始人Herbert Freeman命名(Freeman, 1961)。图\ref{fig:1.6.1} 显示了定义从一个像素到其邻居路径上的八个可能方向的值。本例中使用的像素阵列如图\ref{fig:1.6.2}a所示;图\ref{fig:1.6.2}b显示了示例程序的输出。算法从样本$x = 3$, $y = 2$开始,寻找$\text{高程}= 200$的等高线边缘并跟踪。

\begin{figure}[htbp]%[htbp]
  \centering
  \includegraphics[totalheight=1.8in]{./fig/1.6.1.png}
  \caption{从点P到它的八个邻居的方向向量。} 
  \label{fig:1.6.1}
\end{figure}

\begin{figure}[htbp]%[htbp]
  \centering
  \includegraphics[totalheight=3in]{./fig/1.6.2.png}
  \caption{a)标高数据采样样例。\\b)绘制$\text{高程}= 200$等高线轮廓的22个点向量。} 
  \label{fig:1.6.2}
\end{figure}

该算法的核心在于知道如何以顺时针方向沿着轮廓的方式选择邻居。测试图\ref{fig:1.6.2}a,并想象您从$x = 1$, $y = 2$的采样点出发,沿着高程= 200的等高线边缘行走。为了顺时针移动轮廓线,你的第一个移动应该是$x = 1$, $y = 3$处的样本。你走在自己轮廓的边缘;在你的左边是危险的悬崖,它向下坠落到一个较低的高度。当你沿着轮廓走的时候,注意你的头部是变化的,但是悬崖总是在你的左边。在选择下一步的时候,你总是试着向前走,向左走,而不是走下悬崖。如果你不能向前移动到你的左边,你试着顺时针方向:直走。如果不行,你试着顺时针方向向前,向右,等等。如果你发现自己在一个海角上,你顺时针方向的方向会让你掉头,重新走一段路。最终,你将完全沿着轮廓移动,然后回到你的起点。

算法的工作原理是一样的。\textit{build()}过程构建围绕轮廓边缘移动时所采取的方向的Freeman链。\textit{build()}调用\textit{neighbor()}过程来获取路径上的下一个邻居。\textit{neighbor()}反过来调用\textit{probe()}来查找该邻居。最低级的过程是\textit{in\_cont()},它只是简单地测试给定的样本是否在轮廓中。请注意,采样数据的整个数组不需要立即放入内存;\textit{in\_cont()}可以修改为随机访问脱机存储,如果需要的话。

\textit{neighbor()}中的\textit{last\_dir}变量维护\textit{neighbor()}的方向感。检查图3,看看邻居()过程是如何实现上面描述的“尝试向前移动并向左移动”的步骤的。假设你从样本a到达样本P,那么\textit{last\_dir}是2,而样本C总是在等高线之外,所以第一个探测的邻居是D。从P到D的方向是3;\textit{new\_dir←last\_dir + 1}。

\begin{figure}[htbp]%[htbp]
  \centering
  \includegraphics[totalheight=1.5in]{./fig/1.6.3.png}
  \caption{从P点通过A或B点移动到D点} 
  \label{fig:1.6.3}
\end{figure}

现在假设您从b到达了P, \textit{last\_dir}是1,C仍然在等高线之外,D仍然是要探测的第一个邻居。P到D的方向仍然是3;\textit{new\_dir←last\_dir + 2}。

请注意,当\textit{last\_dir}等于0、2、4或6时,到达P点的情况都是全等的(它们只是被旋转了90度)。类似地,\textit{last\_dir} = 1、3、5和7的情况都相等。因此,\textit{neighbor()}使用的简单规则是将\textit{new\_dir}设置为\textit{last\_dir + 1}(如果\textit{last\_dir}为偶数),或将其设置为+ 2(如果\textit{last\_dir}为奇数)。当然,\textit{new\_dir}的范围必须保持在0到7之间,因此加法对8取模。

E的取值范围从0到7,所以加法对8取模。唯一剩下的微妙之处是如何正确地选择第一个移动,以便以顺时针方向围绕轮廓开始。这是很容易实现的:算法,当给定一个起点在轮廓的任何地方,在轮廓向左移动,直到它遇到边缘。这保证了路径从边缘开始。它还保证了初始排列如图3所示:路径从样本P开始,C处的邻居不在等高线中,并且\textit{new\_dir}的值应为3。这意味着\textit{last\_dir}的初始值应该是1。算法在\textit{build()}过程中设置它。

这个示例程序是为了演示Freeman链背后的思想而编写的,但它的存储效率并不高。链表中的每个成员占用一个整数和一个指针。但正如Freeman在他最初的工作中指出的,每个方向向量编码只需要三个比特。使用三位的实现将使用一个大的内存块而不是链表,并且将有将方向向量链打包到块中的过程,并按顺序提取它们。为了确定应该为保存轮廓中所有方向向量的块分配多少内存,将使用轮廓跟踪算法的简化版本。它会沿着轮廓线计算描述完整路径所需的方向向量的数量,但它不会存储方向向量。一旦确定了矢量的数量,就会分配内存,调用主算法重描轮廓,并将方向矢量打包到内存块中。

前面的方法比示例程序更有效,但它用速度换取了内存空间。第三种方法仍然允许轮廓具有任意长度,同时有效地使用内存空间,同时保持良好的速度。它通过消除预扫描步骤来实现这一点。对于使用大数据集或不保存在内存中的数据集的实现来说,这是一个重要的考虑因素。方法是使用一个简单的链表,就像示例程序一样。但是,列表中的每个成员都有一个分配的内存块,而不是一个整数。这个块包含许多方向向量,每一个都被压缩成3个比特。更多的数据块将被分配,并根据需要链接到列表中,因为轮廓是遵循的。包装和提取带菌者将需要程序,而且必须保持额外的管理资料,以便使一切都处于控制之下。这种技术为链表中的指针使用了少量空间,但仍然比示例程序的内存效率高得多。这种方法的权衡是在实际编程中经常遇到的问题:为了在保持速度的同时节省存储空间,会增加程序的复杂性。

最后,一些实现可能根本不需要在内存中保存轮廓的表示;它们可以简单地将方向向量写入顺序访问磁盘文件或某些输出设备或并发进程。在这种情况下,示例程序的\textit{build()}过程将被修改。

See also G3, A.5.

\newpage
\section{合成黑白位图}
\begin{center}
\small{
David Salesin   \hspace{25ex}                 Ronen Barzel\\
\hspace{10ex} Cornell University    \hspace{5ex}        and  \hspace{3ex}   California Institute of Technology\\
\hspace{3ex} Ithaca, New York           \hspace{20ex}           Pasadena, California}
\end{center}

\subsection*{介绍}

典型的位图编码黑白像素。添加一个辅助位图允许我们表示透明的像素。这种两位表示对于非矩形或有孔的黑白图像很有用。它还为组合位图提供了一组更丰富的操作。我们用布尔值对$(\alpha, \beta)$编码三个可能的像素值,如下所示:



\begin{center}
\begin{tabular}{ccl}
\hline
$\alpha$ & $\beta$ & Meaning \\
\hline
1 & 0 & Black \\
1 & 1 & vvnite \\
0 & 0 & Transparent \\
0 & 1 & Undefined \\
\hline
\end{tabular}
\end{center}


\subsection*{Compositing Bitmaps}

我们可以使用合成操作$\mathrm{R} \leftarrow \mathrm{P}\quad \mathbf{op} \quad \mathrm{Q}$
将两个像素$\mathrm{P}=\left(\mathrm{P}_{\alpha^{\prime}}, \mathrm{P}_{\beta}\right)$
和$\mathrm{Q}=\left(\mathrm{Q}_{\alpha^{\prime}} \mathrm{Q}_{\beta}\right)$
组合成一个新的像素$\mathrm{R}=\left(\mathrm{R}_{\alpha^{\prime}} \mathrm{R}_{\beta}\right)$,如表一所示。


该表是将全彩合成代数(Porter and Duff, 1984)简化为两个位域(Salesin and Barzel, 1986)。注意,表中$\mathrm{R}_{\alpha}$和$\mathrm{R}_{\beta}$的方程现在是布尔公式:and被写成乘法,OR写成加法,XOR写成$\oplus$。布尔操作可以使用一系列标准的“bitblt”操作一次对整个位图执行。根据操作的不同,所需的bitblt总数从2到4不等。



\begin{table}[htbp]
\begin{center}
\includegraphics[max width=4in]{2022_11_30_0cbb01a33d99487fc27fg-066}
\end{center}
\caption{位图合成操作}
\end{table}

表中的数字描述了操作对两个示例位图$\mathrm{P}$和q的影响。在这些数字中,使用灰色调表示结果中的透明区域。


over运算符用于将非矩形黑白位图置于现有图像之上——它非常适合绘制游标。in和out操作符允许一个位图作为另一个位图的哑光—例如,如果$P$是一个“砖”纹理,而Q是一个“建筑”,那么P in Q就会用砖对建筑进行贴片。如果一个位图同时充当另一个位图的哑光和背景,那么atop操作符很有用——例如,它允许将一小块纹理增量地绘制到现有位图上。


See also G1, 210.


\newpage

\section{计算机动画的$2\frac{1}{2}D$景深模拟}

\begin{center}
\small{
Cary Scofield  \\
Hewlett-Packard Company\\
Chelmsford, Massachusetts           }
\end{center}

\subsection*{介绍}

景深被定义为在光学透镜系统中包围焦平面的区域,在该区域内物体保持确定的焦质量。长期以来,摄影师和电影摄影师一直利用相机镜头和光圈的这一方面,将观众的注意力从感兴趣的领域以外的区域引导到图像的特定部分。正因为如此,在计算机动画系统中加入景深效果是非常有利的。

这个gem描述了一个$2 \frac{1}{2}$-D的景深算法,用于模拟计算机生成的动画中的焦点变化。这种特殊的算法实际上独立于任何隐藏面去除技术。我们根据深度将3d场景分层为独立渲染的不相交对象组。生成的图像经过滤波模拟景深,然后通过合成后处理重新组合成单幅图像。当逐渐改变滤镜应用于动画序列的连续帧时,其效果是将观看者的注意力从场景的一个深度面拉到另一个深度面。


\subsection*{相关工作}
以前模拟景深的尝试涉及使用针孔相机模型(Potmesil和Chakravarty, 1982)。然而,该算法没有考虑到透镜表面提供了连续的环境不同视图这一事实。分布射线追踪(Cook et al., 1984)克服了这一缺陷,但将该技术嵌入到渲染过程中。我们的算法可以被认为是这两种方法的折中:虽然我们不整合阴影和景深,但我们确实以一种受限的方式结合了表面可见性和景深。


\subsection*{算法}
我们的方法本质上是一个三个阶段的过程:

\begin{enumerate}
	\item 隐藏表面去除阶段:\\
	在第一个阶段中,可以手动或自动地将对象分层或聚类到前景和背景组中。每个簇或组被单独渲染成它自己的图像,每个像素的alpha通道有一个不透明度掩码(见图1,颜色插入)。这种不透明度遮罩最终在第三阶段发挥作用,但它可以在第二阶段进行修改。

	\item 滤波器后处理阶段:\\
	该阶段使用类似于指数低通滤波器的卷积掩模来模糊各种图像,以模拟景深效果。由于我们的算法与模拟透镜的焦距无关,这允许我们自由地操纵图像的模糊程度。然而,我们的目的是创造一个现实的效果,所以我们仔细选择模糊程度。还必须小心避免“渐晕”(Perlin, 1985),如果过滤算法不补偿图像边界之外的未知区域,就会出现这种现象。这是通过重新计算加权滤波器,每当卷积掩模的任何部分被图像边界剪辑。
	
	\item 合成阶段:\\
	最后,在这一阶段,我们遵循波特和达夫(1984)、达夫(1985)和麦克斯和勒纳(1985)建立的算法。第一阶段中不透明度蒙版的重要性在这里发挥了作用,因为它允许我们在将前景图像叠加到背景图像时避免混淆工件。

\end{enumerate}

\subsection*{焦距变化模拟}
正如在介绍中所述,电影摄影师长期以来一直使用焦点的变化来推动或拉观众的注意力从场景的一个部分到另一个部分(例如,从前景到背景)。这种“聚焦”是未经过滤的前景对象(“聚焦中”)与模糊的背景(“超出”焦平面)叠加的结果。因此,将观众的注意力从前景拉到背景,相当于在一系列帧中,将镜头的“焦平面”沿着光轴从前景逐渐转移到背景。因为我们不使用镜头和光圈相机模型,我们必须从一帧到一帧修改模糊滤镜的形状。假设发生焦点变化模拟的帧数是先验的,我们可以很容易地对滤波器的“模糊度”进行插值,范围从一个很小的值到一个能给我们期望的最大图像模糊度的大小。这是为前景对象做的。对于相应的背景图像,我们使用相同的插值值系列,只是顺序相反。当过滤过程完成后,两个独立的前景和背景图像流在最后一个阶段被拼接在一起,形成拍摄的最终图像帧。图b、c和d(见颜色插入)分别是由这个过程产生的动画序列的第一帧、中间帧和最后帧的例子。作为一个边注,这一过程所获得的结果与塞尔动画中偶尔使用的多平面相机系统所获得的结果非常相似(Thomas和Johnston, 198l)。


\subsection*{致谢和历史记录}
这本书是几年前与詹姆斯·阿尔沃(James Alvo)合作撰写的一篇更详细但从未发表过的论文的浓缩。作为一个历史笔记,在这个宝石中描述的焦点变化模拟被用于两个阿波罗计算机射线追踪动画公司中的第一个(即“Quest”)。

\begin{figure}
\centering
	\subfigure[由它们自己渲染的场景中的一簇物体。]{\includegraphics[max width=.35\columnwidth]{2022_11_30_0cbb01a33d99487fc27fg-070}}\hspace{5pt}
	\subfigure[前景和背景物体的合成图像。前景是“聚焦”。]{\includegraphics[width=.35\columnwidth]{2022_11_30_0cbb01a33d99487fc27fg-070(2)}}\\
	\subfigure[合成图像。前景和背景对象模糊化。]{\includegraphics[width=.35\columnwidth]{2022_11_30_0cbb01a33d99487fc27fg-070(1)}}\hspace{5pt}
	\subfigure[合成图像。动画序列的最后一帧,背景对象“聚焦”。]{\includegraphics[width=.35\columnwidth]{2022_11_30_0cbb01a33d99487fc27fg-070(3)}}
\end{figure}

\newpage
\section{复合区域的快速边界生成器}
\begin{center}
\small{
Eric Furman  \\
General Dynamics, Electronics Division\\
San Diego, California           }
\end{center}

\subsection*{问题}
在计算机图形学的许多应用中,寻找由封闭边界定义的多个区域的轮廓是一个常见的问题。通过求一组圆的包络线可以确定一组雷达站点的二维覆盖,如图1所示。另一个例子是土地利用和城市规划中若干分区的大纲。一般来说,这类应用程序都是将兴趣区域的联合可视化。需要一种算法来找到这些区域的复合边界或包络线。换句话说,我们希望显示一组二维区域的轮廓。每个区域在二维上都是一个封闭边界。常见的区域原语是圆、多边形和椭圆,但任何其他封闭区域轮廓都可以使用所描述的技术很好地工作。在这个gem中,我们将使用圆作为我们的基本示例区域。图1到图3显示了该算法的步骤。圆集合的轮廓有许多短的连接弧或扇形。


\subsection*{其他方法}
针对多雷达站点问题的几种解决方案都采用了直接分析的方法。他们通过一系列雷达范围圆进行工作,通过将每个新圆与以前的交叉点生成的弧相交,从而创建一组弧(Bowyer and Woodwark, 1983)。不相交的圆必须带着,并与每个新圆相交。对每个新圆进行内部/外部测试,并修改弧集以删除一些弧段并添加新的弧。不幸的是,随着圆列表的增长,生成边界的时间随着圆数量的平方而增加。已经实现了一些改进,例如完全丢弃包含在其他圆中的圆,并使用边界框测试来限制更昂贵的圆/弧相交测试的数量。

\begin{figure}
\begin{center}
\includegraphics[max width=\textwidth]{2022_11_30_0cbb01a33d99487fc27fg-072}
\end{center}
\caption{A set of circles shown in a frame buffer.}
\end{figure}

\begin{figure}
\begin{center}
\includegraphics[max width=\textwidth]{2022_11_30_0cbb01a33d99487fc27fg-072(1)}
\end{center}
\caption{The circles of Fig. 1 after filling.}
\end{figure}


Figure 2. The circles of Fig. 1 after filling.

\begin{figure}
\begin{center}
\includegraphics[max width=\textwidth]{2022_11_30_0cbb01a33d99487fc27fg-073}
\end{center}
\caption{Scallops. The envelope of Fig. 1 circles.}
\end{figure}


这种直接的分析方法适合于创建分辨率远高于典型帧缓冲区所能显示的区域边界数据集。然而,可视化一组区域的包络可以更快地完成。


\subsection*{快速生成边界}
一种混合了计算机图形学和图像处理技术的快速生成区域边界的方法。该算法由两个基本步骤组成,用于帧缓冲区的有限分辨率。首先,对于每个封闭原语,绘制填充区域并将其剪切到帧缓冲区限制。尽管我们希望只显示许多闭合对象的信封,但我们同时绘制和填充每个对象。通常绘图和填充可以通过精心的算法构造在同一个步骤中完成。其次,在整个帧缓冲区上应用侵蚀运算符来删除所有填充的内部点。没有填充的对象,侵蚀不会使它们减少到复合边界。这个过程如图1-3所示,每个圆都是独立绘制和填充的。


该算法的执行时间是所有区域的绘图和填充时间加上单道侵蚀操作的时间。在我们的示例中,这是绘制/填充所有圆所需的时间加上侵蚀它们所需的时间。当绘制单个重叠对象时,不填充先前由另一个区域填充的像素似乎可以节省一些时间。不幸的是,测试确定一个区域是否已经被填满通常比仅仅将数据写入帧缓冲区要花费更长的时间。所需的时间与所有区域的填充面积成线性增长,对于直接分析方法的平方增长率有相当大的改进。

附录II中用于生成填充圆的示例$\mathrm{C}$代码改变了计算机图形文本中给出的通常的中点圆生成算法,以创建填充圆(Foley et al., 1990)。通过调用raster\_ fill函数来填充圆,该函数直接实现了使用fill\_circle 函数中的二阶偏差生成中点圆的算法。


将填充区域侵蚀到它们的边界轮廓,查看帧缓冲区中每个像素及其最近的四个连接邻居。一个填充的像素值或颜色将被擦除为侵蚀值,只有当它的所有四个连接的邻居像素也被填充相同的值。对背景值的侵蚀将只留下以原始填充颜色绘制的边界像素。然而,侵蚀到不同的颜色将留下轮廓在原始填充值和填充内部用新的侵蚀值。在这个测试过程中使用了三个光栅缓冲区,以避免在以后计算其他像素时替换帧缓冲区中的像素,或者需要从一个帧缓冲区渲染到另一个帧缓冲区。在附录II的$\mathrm{C}$代码中,栅格缓冲区向帧缓冲区添加了一个单像素的边界。用对象的原始像素填充值填充这个边界将留下一个开放的边缘,在这个边界被剪切在帧缓冲区的边缘。当边界设置为背景值(在代码中为零)时,在发生帧边缘裁剪的地方绘制一个封闭边缘包络线。这个过程是图像处理的二值边缘检测(Gonzalez和Wintz, 1987)在计算机图形学中的一个相当简单的应用。

\subsection*{注意事项}
尽管这种技术简单快速,像大多数帧缓冲算法一样,它只能精确到最近的像素。解析解的精度将受到计算所用的算术精度的限制。


对于圆,可以通过生成一个固定圆的栅格跨度宽度表来提高速度。直径应该至少是帧缓冲区最大尺寸宽度的两倍,以保持在Nyquist区间内。然后,该表可用于任何所需圆大小的比例宽度查找。表查找消除了进一步的圆生成,但仍然必须执行剪切和填充操作。对于任何直接可伸缩和未旋转的区域,也可以获得类似的表好处。


为了帮助用户验证函数的正确实现,$\mathrm{C}$代码包含一个简单的程序来测试填充和侵蚀例程。一个很小的伪帧缓冲区使用了用简单的"printf"语句在ASCII中显示的左上角。











\end{CJK}
